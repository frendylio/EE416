\documentclass[11pt]{article}
\usepackage{scribe}
\usepackage{graphicx}

% Uncomment the appropriate line
%\Scribe{Your name}

\Scribes{Frendy Lio Can}
\LectureDate{October 23, 2020}
\LectureTitle{Homework Assignment \#4}

%\usepackage[mathcal]{euscript}


\begin{document}

\MakeScribeTop

%\paragraph{This is a paragraph heading} Paragraph.

%%%%%%%%%%%%%%%%%%%%%%%%%%%%%%%%
% PROBLEM 1 A
%%%%%%%%%%%%%%%%%%%%%%%%%%%%%%%%
\paragraph{\noindent\textbf{\LARGE{Problem 1, a)}}}

\begin{flushleft}
    We know that:
\end{flushleft} 
\begin{equation*}
\begin{split}
    x[n] \rightarrow & \uparrow U \rightarrow H(e^{i\omega}) \rightarrow y[n] \\
    & Therefore \\
    x[m,n] \rightarrow & \uparrow 2 \rightarrow H(e^{i\mu}, e^{i\nu}) \rightarrow y[n] \\
\end{split}
\end{equation*}

\begin{equation*}
\begin{split}
    h[m,n] = & h[-k, -l] = 
    \begin{bmatrix}
        0.25 & 0.5 & 0.25 \\ 
        0.5 & 1 & 0.5 \\
        0.25 & 0.5 & 0.25
    \end{bmatrix} \\
    x_2[m,n] = & 
    \begin{bmatrix}
        0 & 0 & 1 & 0 & 1 \\
        0 & 0 & 0 & 0 & 0 \\ 
        0 & 0 & 1 & 0 & 1 \\
        0 & 0 & 0 & 0 & 0 \\ 
        0 & 0 & 0 & 0 & 0
    \end{bmatrix}   \\
    y[m,n] = &   
    \begin{bmatrix}
        0 & 0.5 & 1 & 1 & 1 \\
        0 & 0.5 & 1 & 1 & 1 \\ 
        0 & 0.5 & 1 & 1 & 1 \\
        0 & 0.25 & 0.5 & 0.5 & 0.5 \\ 
        0 & 0 & 0 & 0 & 0
    \end{bmatrix}     
\end{split}
\end{equation*}

%%%%%%%%%%%%%%%%%%%%%%%%%%%%%%%%
% PROBLEM 1 B
%%%%%%%%%%%%%%%%%%%%%%%%%%%%%%%%
\paragraph{\noindent\textbf{\LARGE{Problem 1, b)}}}  
\begin{equation*}
\begin{split}
H(e^{i\mu}, e^{i\nu}) = & \sum_{m,n = -1}^1 0.5^{|m+n|} e^{-i(\mu m + \nu n)} \\
    = & \sum_{m = -1}^1 0.5^{|m|} e^{-i(\mu m)} \sum_{n = -1}^1 0.5^{|n|} e^{-i(\nu n)} \\
    = & (0.5e^{i\mu} + 1 + 0.5e^{-i\mu})(0.5e^{i\nu} + 1 + 0.5e^{-i\nu}) \\
    = & (1 + cos \mu) (1 + cos\nu)
\end{split}
\end{equation*}

%%%%%%%%%%%%%%%%%%%%%%%%%%%%%%%%
% PROBLEM 1 C
%%%%%%%%%%%%%%%%%%%%%%%%%%%%%%%%
\paragraph{\noindent\textbf{\LARGE{Problem 1, c)}}}  
\begin{equation*}
\begin{split}
Y(e^{i\mu}, e^{i\nu}) = & \sum_{k,l = 0}^1 H(e^{\frac{i(\mu - 2\pi k)}{2}}, e^{\frac{i(\nu - 2\pi l)}{2}}) 
X(e^{\frac{i(\mu - 2\pi k)}{2}}, e^{\frac{i(\nu - 2\pi l)}{2}})
\end{split}
\end{equation*}

%%%%%%%%%%%%%%%%%%%%%%%%%%%%%%%%
% PROBLEM 1 D
%%%%%%%%%%%%%%%%%%%%%%%%%%%%%%%%
\paragraph{\noindent\textbf{\LARGE{Problem 1, d)}}}  
  
\begin{flushleft}
The advantages of interpolation methods is that they are simple since they are used to simplify our signals by sampling them.
It also improves anti-aliasing filter performance and reduces noise. 
\newline
\newline
The disadvantages of interpolation method is that we require a higher sample rate in order to have a good resolution conversion.
\end{flushleft} 
    





\end{document}
