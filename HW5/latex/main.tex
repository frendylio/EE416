\documentclass[11pt]{article}
\usepackage{scribe}
\usepackage{graphicx}

% Uncomment the appropriate line
%\Scribe{Your name}

\Scribes{Frendy Lio Can}
\LectureDate{November 3, 2020}
\LectureTitle{Homework Assignment \#5}

%\usepackage[mathcal]{euscript}


\begin{document}

\MakeScribeTop

%\paragraph{This is a paragraph heading} Paragraph.

%%%%%%%%%%%%%%%%%%%%%%%%%%%%%%%%
% PROBLEM 1 A
%%%%%%%%%%%%%%%%%%%%%%%%%%%%%%%%
\paragraph{\noindent\textbf{\LARGE{Problem 1, a)}}}

\begin{equation*}
\begin{split}
    T(x) = & F^{-1}_d (F_X(x)) \\
    F_d(x) = & \int \frac{1}{(L-1)-0} (L-1) dx \\
            = & x \\
    F_d^{-1}(x) = & x \\
    F_X(x) = & \int_0^{x} \frac{2x}{L - 1} * (L-1) dx \\
            = & x^2 \rvert_0^{x} \\ 
            = & x^2 \\
    F^{-1}_d (F_X(x)) = & x^2
\end{split}
\end{equation*}
\begin{flushleft}
    Therefore, $T(x) = x^2$.
\end{flushleft} 


%%%%%%%%%%%%%%%%%%%%%%%%%%%%%%%%
% PROBLEM 1 B
%%%%%%%%%%%%%%%%%%%%%%%%%%%%%%%%
\paragraph{\noindent\textbf{\LARGE{Problem 1, b)}}}  
\begin{equation*}
\begin{split}
    T^\prime(y) = & F^{-1}_d (F_Y(y)) \\
    F_d(x)  = & \int \frac{3z^2}{(L-1)^3} * (L -1) dz \\
            = & \frac{1}{(L - 1)^2} *  z^3 \\ 
            = & \frac{1}{(L - 1)^2} *  z^3 \\
    F_d^{-1}(x) = & \sqrt[3]{x (L-1)^2 } \\
    F_X(x) = & \int_0^{x} \frac{1}{L-1} * (L-1) dx \\
            = & x \\
    F^{-1}_d (F_X(x)) = & \sqrt[3]{x (L-1)^2 } 
\end{split}
\end{equation*}
\begin{flushleft}
    Therefore, $T^\prime(y) =(L-1) \sqrt[3]{y (L-1)^2 }$.
\end{flushleft} 
        
%%%%%%%%%%%%%%%%%%%%%%%%%%%%%%%%
% PROBLEM 1 C
%%%%%%%%%%%%%%%%%%%%%%%%%%%%%%%%
\paragraph{\noindent\textbf{\LARGE{Problem 1, c)}}}  
\begin{equation*}
\begin{split}
    f(x) = & \frac{2x}{L-1} \\
    F(x) = & \int_0^{x} \frac{2x}{L-1} dx \\
         = & \frac{1}{L-1}x^2 \\
    f(z) = &\frac{3z^2}{(L-1)^3} \\
    F_d(x) = &  \int  \frac{3x^2}{(L-1)^3}* (L-1) dx \\
            = & \frac{x^3}{(L-1)^2} \\
    F_d^{-1}(x) = & \sqrt[3]{x*(L-1)^2 } \\
    \tilde{T}(x) = & F^{-1}_d (F_X(x)) \\
                = & \sqrt[3]{(L-1) x^2} 
\end{split}
\end{equation*}
\begin{flushleft}
    Therefore, $z = cT^\prime(x) = c\sqrt[3]{(L-1) x^2}  $.
\end{flushleft}  

%%%%%%%%%%%%%%%%%%%%%%%%%%%%%%%%
% PROBLEM 2 a
%%%%%%%%%%%%%%%%%%%%%%%%%%%%%%%%
\paragraph{\noindent\textbf{\LARGE{Problem 2, a)}}}  
  
\begin{equation*}
\begin{split}
    y = & \theta \\ 
    \frac{\partial }{\partial \theta} \sum_{n = 1}^{N} \rho(x_n - \theta) = & 0 \\
    \frac{\partial }{\partial \theta} \sum_{n = 1}^{N} (x_n - \theta)^2 = & 0 \\
    \sum_{n = 1}^{N} -2(x_n - \theta) = & 0 \\
    \sum_{n = 1}^{N} (x_n) - \sum_{n = 1}^{N} \theta=  & 0 \\
    \sum_{n = 1}^{N} (x_n) - n \theta=  & 0 \\
    N \theta = & \sum_{n = 1}^{N} (x_n) \\
    \theta = & \frac{1}{N} \sum_{n = 1}^{N} (x_n) 
\end{split}
\end{equation*}
    
\begin{flushleft}
    Therefore, $y = \theta =  \frac{1}{N} \sum_{n = 1}^{N} (x_n) $, which is the mean value.
\end{flushleft}  

%%%%%%%%%%%%%%%%%%%%%%%%%%%%%%%%
% PROBLEM 2 b
%%%%%%%%%%%%%%%%%%%%%%%%%%%%%%%%
\paragraph{\noindent\textbf{\LARGE{Problem 2, b)}}}  
  
\begin{equation*}
\begin{split}
    y = & \theta \\ 
    \frac{\partial }{\partial \theta} \sum_{n = 1}^{N} \rho(x_n - \theta) = & 0 \\
    \frac{\partial }{\partial \theta} \sum_{n = 1}^{N} |x_n - \theta| = & 0 \\
        = & \sum_{n = 1}^{N} sign(x_n - \theta) \\
\end{split}
\end{equation*}
    
\begin{flushleft}
    Therefore, $y = \theta =  \sum_{n = 1}^{N} sign(x_n - \theta) $, which is the median.
\end{flushleft}  

%%%%%%%%%%%%%%%%%%%%%%%%%%%%%%%%
% PROBLEM 2 c
%%%%%%%%%%%%%%%%%%%%%%%%%%%%%%%%
\paragraph{\noindent\textbf{\LARGE{Problem 2, c)}}}  
  
\begin{equation*}
\begin{split}
    y = & \theta \\ 
    \frac{\partial }{\partial \theta} \sum_{n = 1}^{N} \rho(x_n - \theta) = & 0 \\
    \frac{\partial }{\partial \theta} \sum_{n = 1}^{N} |x_n - \theta|^{0.5} = & 0 \\
        = & 0.5 \sum_{n = 1}^{N} sign(x_n - \theta) |x_n - \theta| ^{-0.5}
\end{split}
\end{equation*}
    
\begin{flushleft}
    Therefore, $y = root_\theta\{0.5 \sum_{n = 1}^{N} sign(x_n - \theta) |x_n - \theta| ^{-0.5}\}$.
\end{flushleft} 

%%%%%%%%%%%%%%%%%%%%%%%%%%%%%%%%
% PROBLEM 2 d
%%%%%%%%%%%%%%%%%%%%%%%%%%%%%%%%
\paragraph{\noindent\textbf{\LARGE{Problem 2, d)}}}  
  
\begin{flushleft}
    No, it is not linear because of the $|x_n - \theta|$.
    It is not homogeneous as well because of the $|x_n - \theta|$.

\end{flushleft}

\end{document}
